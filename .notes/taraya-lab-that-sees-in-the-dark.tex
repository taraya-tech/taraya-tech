\documentclass[12pt]{article}
\usepackage[T1]{fontenc}
\usepackage[margin=1in]{geometry}
\usepackage{microtype}
\usepackage{setspace}
\usepackage{titlesec}
\usepackage{lmodern}
\usepackage{parskip}
\usepackage{graphicx}
\usepackage{hyperref}

\titleformat{\section}{\large\bfseries}{\thesection}{1em}{}
\titleformat{\subsection}{\normalsize\bfseries}{\thesubsection}{1em}{}

\title{\vspace{-2cm}Taraya: The Lab That Sees in the Dark\vspace{1cm}}
\author{}
\date{}

\begin{document}
    \maketitle
    \begin{spacing}{1.25}

        \section*{Taraya: The Lab That Sees in the Dark}

        Today, Taraya is building the world’s first \textbf{global, virtual, AI-assisted lab for life sciences}.

        It simulates \textbf{protein folding} with atomic precision. It models \textbf{drug efficacy}, \textbf{long-tail side effects}, and \textbf{biochemical interactions} — in silico, before a molecule is ever made. This lab doesn’t just accelerate discovery — it redefines the cycle from hypothesis to breakthrough.

        Powered by a sovereign compute grid, Taraya’s infrastructure is secure, deterministic, and self-governing. It doesn’t depend on foreign fabs or fragile supply chains. It hosts simulation workflows, safely, at planetary scale.

        But this story doesn’t begin with cloud infrastructure or artificial intelligence.

        It begins in a darkroom. With a teenager. In 1980.

        \section*{The First Virtual Lab}

        At age 14, the youngest paid researcher at the University of Virginia School of Medicine was handed a challenge few adults could solve: automate a 100\% darkroom-based vision research lab filled with legacy analog instruments.

        The lab investigated photoreceptors, retinal chemistry, and neural signaling. It used oscilloscopes, photometers, spectrophotometers, HPLCs — all analog, with no digital interfaces.

        So he created them.

        \begin{itemize}
            \item Disabled status lamps and pilot bulbs to maintain darkness.
            \item Built latching TTL circuits to read machine state.
            \item Used relays and D/A converters to override knobs and switches.
            \item Wrote GPIB drivers in \texttt{6502 assembly} to control and monitor everything.
            \item Connected the system to S-100 and PET-class machines for orchestration.
        \end{itemize}

        He created a programmable control plane before that term existed. And because the lab had no light?

        He gave it a voice.

        He designed a \textbf{speaker-independent, pitch-independent voice recognition system} using breath-pause detection and vector-quantized matching. The command vocabulary had 70 words. Audio feedback was delivered via synthesized and prerecorded voice output.

        \textit{“Protocol complete.”}
        \textit{“Setpoint adjusted.”}
        \textit{“Calcium trace recording.”}

        It wasn’t a toy. It was a working system — one that drove real research and appeared in peer-reviewed literature. A published experiment in \textit{Analytical Biochemistry} cited hardware and software he designed.

        Then the engineers started showing up.

        From Tektronix. From Beckman. From Waters, Bio-Rad, Perkin-Elmer, and HP. They came to see what their own manuals said wasn’t possible.

        How had a teenager reverse engineered their hardware, written custom software, and networked it all into an autonomous, audio-controlled research lab?

        \section*{45 Years Later: Taraya}

        That same mind is now building \textbf{Taraya}.

        Not a single lab. A \textbf{civilization-scale compute city}. A sovereign infrastructure stack where biology, simulation, and intelligence converge.

        Where protein folding is simulated across datacenter villages.
        Where drug pipelines are de-risked in days, not decades.
        Where the darkroom becomes a continent, and the experiment never sleeps.

        \section*{How does a polymath live their arc?}

        By solving the same problem across decades —
        through atoms and logic, wet labs and simulation grids —
        and scaling each solution to its sovereign form.

        Not once. But again.

    \end{spacing}
\end{document}
